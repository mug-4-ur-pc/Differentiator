\documentclass[a4paper,oneside,final,12pt,russian]{extarticle}
\usepackage[utf8]{inputenc}
\usepackage[T2A,T1]{fontenc}
\usepackage[russian]{babel}
\usepackage{vmargin}
\usepackage{indentfirst}
\usepackage{amsmath}
\usepackage{amsfonts}
\usepackage{hyperref}
\usepackage{cmap}
\usepackage{listings}
\usepackage{ulem}
\usepackage{breqn}
\setpapersize{A4}
\setmarginsrb{2cm}{1.5cm}{1cm}{1.5cm}{0pt}{0mm}{0pt}{13mm}

\usepackage{blindtext}
\usepackage{setspace}
\sloppy
\hyphenpenalty=5000

\usepackage{titling}
\setlength{\droptitle}{-1cm}

\title{\normalsize \bf РЕШЕНИЕ СЛОЖНЫХ ЗАДАЧ \\ НАХОЖДЕНИЕ ПРОИЗВОДНОЙ И ФОРМУЛЫ ТЕЙЛОРА}
\author{\small Иванов Денис \\ Б05--132}

\newcommand{\e}{\mathsf{e}}
\begin{document}

\maketitle

Для разложения выражения $\operatorname{cos}(2.00 \cdot x ^{2.00 } ) + \operatorname{f}(\e ^{x } ) $ в ряд Тейлора нам нужно найти $3$ первых производных. Ну что ж, давайте сделаем это!
По крайней мере попробуем)

\section{}
Давайте возьмём производную $\operatorname{cos}(2.00 \cdot x ^{2.00 } ) + \operatorname{f}(\e ^{x } ) $. Во имя котиков мы обязаны сделать это!

Нарисуем кишечную палочку
\begin{dmath*}
(x )' = 1.00 .
\end{dmath*}
Ах если бы сложение стало делением, \\ 
А, может, даже умножением! \\ 
Вот было бы прекрасно,\\ 
Если бы муссонные ветра дули не напрасно. \\ 
Тогда может быть и был бы смысл в этой жизни,\\ 
И равен бы он был морали этих строк, которой нет...\\ 
Так о чём же это я? Да ни о чём, собственно говоря.
Я просто хочу удалить этот файл с бессмыденным матаном 
и никогда его больше не видеть. :(
\begin{dmath*}
(\e ^{x } )' = \e ^{x } \cdot \operatorname{ln}(\e ) \cdot 1.00 .
\end{dmath*}
Чему равна производная сложной функции?.. А хрен его знает.
\begin{dmath*}
(\operatorname{f}(\e ^{x } ) )' = \operatorname{f}(\e ^{x } ) '\cdot \e ^{x } \cdot \operatorname{ln}(\e ) \cdot 1.00 .
\end{dmath*}
Притворимся, что знаем матан))
\begin{dmath*}
(x )' = 1.00 .
\end{dmath*}
Ах если бы сложение стало делением, \\ 
А, может, даже умножением! \\ 
Вот было бы прекрасно,\\ 
Если бы муссонные ветра дули не напрасно. \\ 
Тогда может быть и был бы смысл в этой жизни,\\ 
И равен бы он был морали этих строк, которой нет...\\ 
Так о чём же это я? Да ни о чём, собственно говоря.
Я просто хочу удалить этот файл с бессмыденным матаном 
и никогда его больше не видеть. :(
\begin{dmath*}
(x ^{2.00 } )' = (2.00 - 1.00 )\cdot x ^{2.00 - 1.00 } \cdot 1.00 .
\end{dmath*}
Тривиальный переход ~--- это всё, на что мы способны
\begin{dmath*}
(2.00 )' = 0.00 .
\end{dmath*}
Легко догадаться, что
\begin{dmath*}
(2.00 \cdot x ^{2.00 } )' = 2.00 \cdot (2.00 - 1.00 )\cdot x ^{2.00 - 1.00 } \cdot 1.00 + 0.00 \cdot x ^{2.00 } .
\end{dmath*}
В СССР дети знали это с детского сада!
\begin{dmath*}
(\operatorname{cos}(2.00 \cdot x ^{2.00 } ) )' = ( -\operatorname{sin}(2.00 \cdot x ^{2.00 } ) ) \cdot (2.00 \cdot (2.00 - 1.00 )\cdot x ^{2.00 - 1.00 } \cdot 1.00 + 0.00 \cdot x ^{2.00 } ).
\end{dmath*}
Легко догадаться, что
\begin{dmath*}
(\operatorname{cos}(2.00 \cdot x ^{2.00 } ) + \operatorname{f}(\e ^{x } ) )' = ( -\operatorname{sin}(2.00 \cdot x ^{2.00 } ) ) \cdot (2.00 \cdot (2.00 - 1.00 )\cdot x ^{2.00 - 1.00 } \cdot 1.00 + 0.00 \cdot x ^{2.00 } )+ \operatorname{f}(\e ^{x } ) '\cdot \e ^{x } \cdot \operatorname{ln}(\e ) \cdot 1.00 .
\end{dmath*}


Благодаря интуиции поймём, что
 \begin{dmath*}
( -\operatorname{sin}(2.00 \cdot x ^{2.00 } ) ) \cdot (2.00 \cdot (2.00 - 1.00 )\cdot x ^{2.00 - 1.00 } \cdot 1.00 + 0.00 \cdot x ^{2.00 } )+ \operatorname{f}(\e ^{x } ) '\cdot \e ^{x } \cdot \operatorname{ln}(\e ) \cdot 1.00  = ( -\operatorname{sin}(2.00 \cdot x ^{2.00 } ) ) \cdot 2.00 \cdot x + \operatorname{f}(\e ^{x } ) '\cdot \e ^{x }  .
\end{dmath*}

Это было очень сложно. Не только лишь все способны 
на это.

\section{}
Давайте возьмём производную $( -\operatorname{sin}(2.00 \cdot x ^{2.00 } ) ) \cdot 2.00 \cdot x + \operatorname{f}(\e ^{x } ) '\cdot \e ^{x } $.

Нарисуем кишечную палочку
\begin{dmath*}
(x )' = 1.00 .
\end{dmath*}
А что если Господь создал этот мир только для того, 
чтобы мы посчитали за него следующую производную:
\begin{dmath*}
(\e ^{x } )' = \e ^{x } \cdot \operatorname{ln}(\e ) \cdot 1.00 .
\end{dmath*}
Притворимся, что знаем матан))
\begin{dmath*}
(x )' = 1.00 .
\end{dmath*}
Легко догадаться, что
\begin{dmath*}
(\e ^{x } )' = \e ^{x } \cdot \operatorname{ln}(\e ) \cdot 1.00 .
\end{dmath*}
Согласно Википедии
\begin{dmath*}
(\operatorname{f}(\e ^{x } ) ')' = \operatorname{f}(\e ^{x } ) ''\cdot \e ^{x } \cdot \operatorname{ln}(\e ) \cdot 1.00 .
\end{dmath*}
Легко догадаться, что
\begin{dmath*}
(\operatorname{f}(\e ^{x } ) '\cdot \e ^{x } )' = \operatorname{f}(\e ^{x } ) '\cdot \e ^{x } \cdot \operatorname{ln}(\e ) \cdot 1.00 + \operatorname{f}(\e ^{x } ) ''\cdot \e ^{x } \cdot \operatorname{ln}(\e ) \cdot 1.00 \cdot \e ^{x } .
\end{dmath*}
Нарисуем кишечную палочку
\begin{dmath*}
(x )' = 1.00 .
\end{dmath*}
Производная любого числа равна числу, не существующему в природе.
\begin{dmath*}
(2.00 )' = 0.00 .
\end{dmath*}
Легко догадаться, что
\begin{dmath*}
(2.00 \cdot x )' = 2.00 \cdot 1.00 + 0.00 \cdot x .
\end{dmath*}
Нарисуем кишечную палочку
\begin{dmath*}
(x )' = 1.00 .
\end{dmath*}
Ах если бы сложение стало делением, \\ 
А, может, даже умножением! \\ 
Вот было бы прекрасно,\\ 
Если бы муссонные ветра дули не напрасно. \\ 
Тогда может быть и был бы смысл в этой жизни,\\ 
И равен бы он был морали этих строк, которой нет...\\ 
Так о чём же это я? Да ни о чём, собственно говоря.
Я просто хочу удалить этот файл с бессмыденным матаном 
и никогда его больше не видеть. :(
\begin{dmath*}
(x ^{2.00 } )' = (2.00 - 1.00 )\cdot x ^{2.00 - 1.00 } \cdot 1.00 .
\end{dmath*}
Производная любого числа равна числу, не существующему в природе.
\begin{dmath*}
(2.00 )' = 0.00 .
\end{dmath*}
Согласно Википедии
\begin{dmath*}
(2.00 \cdot x ^{2.00 } )' = 2.00 \cdot (2.00 - 1.00 )\cdot x ^{2.00 - 1.00 } \cdot 1.00 + 0.00 \cdot x ^{2.00 } .
\end{dmath*}
Раз, два, три, четыре, пять.\\ 
Вышли скобки погулять.\\ 
И поняли, вернувшись домой.\\ 
Производную от аргумента взял кто-то другой.
\begin{dmath*}
(\operatorname{sin}(2.00 \cdot x ^{2.00 } ) )' = \operatorname{cos}(2.00 \cdot x ^{2.00 } ) \cdot (2.00 \cdot (2.00 - 1.00 )\cdot x ^{2.00 - 1.00 } \cdot 1.00 + 0.00 \cdot x ^{2.00 } ).
\end{dmath*}
Ах если бы сложение стало делением, \\ 
А, может, даже умножением! \\ 
Вот было бы прекрасно,\\ 
Если бы муссонные ветра дули не напрасно. \\ 
Тогда может быть и был бы смысл в этой жизни,\\ 
И равен бы он был морали этих строк, которой нет...\\ 
Так о чём же это я? Да ни о чём, собственно говоря.
Я просто хочу удалить этот файл с бессмыденным матаном 
и никогда его больше не видеть. :(
\begin{dmath*}
(-\operatorname{sin}(2.00 \cdot x ^{2.00 } ) )' = -(\operatorname{cos}(2.00 \cdot x ^{2.00 } ) \cdot (2.00 \cdot (2.00 - 1.00 )\cdot x ^{2.00 - 1.00 } \cdot 1.00 + 0.00 \cdot x ^{2.00 } )).
\end{dmath*}
Легко догадаться, что
\begin{dmath*}
(( -\operatorname{sin}(2.00 \cdot x ^{2.00 } ) ) \cdot 2.00 \cdot x )' = ( -\operatorname{sin}(2.00 \cdot x ^{2.00 } ) ) \cdot (2.00 \cdot 1.00 + 0.00 \cdot x )+ ( -(\operatorname{cos}(2.00 \cdot x ^{2.00 } ) \cdot (2.00 \cdot (2.00 - 1.00 )\cdot x ^{2.00 - 1.00 } \cdot 1.00 + 0.00 \cdot x ^{2.00 } ))) \cdot 2.00 \cdot x .
\end{dmath*}
Легко догадаться, что
\begin{dmath*}
(( -\operatorname{sin}(2.00 \cdot x ^{2.00 } ) ) \cdot 2.00 \cdot x + \operatorname{f}(\e ^{x } ) '\cdot \e ^{x } )' = ( -\operatorname{sin}(2.00 \cdot x ^{2.00 } ) ) \cdot (2.00 \cdot 1.00 + 0.00 \cdot x )+ ( -(\operatorname{cos}(2.00 \cdot x ^{2.00 } ) \cdot (2.00 \cdot (2.00 - 1.00 )\cdot x ^{2.00 - 1.00 } \cdot 1.00 + 0.00 \cdot x ^{2.00 } ))) \cdot 2.00 \cdot x + \operatorname{f}(\e ^{x } ) '\cdot \e ^{x } \cdot \operatorname{ln}(\e ) \cdot 1.00 + \operatorname{f}(\e ^{x } ) ''\cdot \e ^{x } \cdot \operatorname{ln}(\e ) \cdot 1.00 \cdot \e ^{x } .
\end{dmath*}


Благодаря интуиции поймём, что
 \begin{dmath*}
( -\operatorname{sin}(2.00 \cdot x ^{2.00 } ) ) \cdot (2.00 \cdot 1.00 + 0.00 \cdot x )+ ( -(\operatorname{cos}(2.00 \cdot x ^{2.00 } ) \cdot (2.00 \cdot (2.00 - 1.00 )\cdot x ^{2.00 - 1.00 } \cdot 1.00 + 0.00 \cdot x ^{2.00 } ))) \cdot 2.00 \cdot x + \operatorname{f}(\e ^{x } ) '\cdot \e ^{x } \cdot \operatorname{ln}(\e ) \cdot 1.00 + \operatorname{f}(\e ^{x } ) ''\cdot \e ^{x } \cdot \operatorname{ln}(\e ) \cdot 1.00 \cdot \e ^{x }  = ( -\operatorname{sin}(2.00 \cdot x ^{2.00 } ) ) \cdot 2.00 + ( -(\operatorname{cos}(2.00 \cdot x ^{2.00 } ) \cdot 2.00 \cdot x )) \cdot 2.00 \cdot x + \operatorname{f}(\e ^{x } ) '\cdot \e ^{x } + \operatorname{f}(\e ^{x } ) ''\cdot \e ^{x } \cdot \e ^{x }  .
\end{dmath*}

Здесь читатель может сделать небольшой перерыв 
и пойти попить айрана.

\section{}
Давайте возьмём производную $( -\operatorname{sin}(2.00 \cdot x ^{2.00 } ) ) \cdot 2.00 + ( -(\operatorname{cos}(2.00 \cdot x ^{2.00 } ) \cdot 2.00 \cdot x )) \cdot 2.00 \cdot x + \operatorname{f}(\e ^{x } ) '\cdot \e ^{x } + \operatorname{f}(\e ^{x } ) ''\cdot \e ^{x } \cdot \e ^{x } $. Во имя котиков мы обязаны сделать это!

Нарисуем кишечную палочку
\begin{dmath*}
(x )' = 1.00 .
\end{dmath*}
Ах если бы сложение стало делением, \\ 
А, может, даже умножением! \\ 
Вот было бы прекрасно,\\ 
Если бы муссонные ветра дули не напрасно. \\ 
Тогда может быть и был бы смысл в этой жизни,\\ 
И равен бы он был морали этих строк, которой нет...\\ 
Так о чём же это я? Да ни о чём, собственно говоря.
Я просто хочу удалить этот файл с бессмыденным матаном 
и никогда его больше не видеть. :(
\begin{dmath*}
(\e ^{x } )' = \e ^{x } \cdot \operatorname{ln}(\e ) \cdot 1.00 .
\end{dmath*}
Нарисуем кишечную палочку
\begin{dmath*}
(x )' = 1.00 .
\end{dmath*}
Легко догадаться, что
\begin{dmath*}
(\e ^{x } )' = \e ^{x } \cdot \operatorname{ln}(\e ) \cdot 1.00 .
\end{dmath*}
Нарисуем кишечную палочку
\begin{dmath*}
(x )' = 1.00 .
\end{dmath*}
Ах если бы сложение стало делением, \\ 
А, может, даже умножением! \\ 
Вот было бы прекрасно,\\ 
Если бы муссонные ветра дули не напрасно. \\ 
Тогда может быть и был бы смысл в этой жизни,\\ 
И равен бы он был морали этих строк, которой нет...\\ 
Так о чём же это я? Да ни о чём, собственно говоря.
Я просто хочу удалить этот файл с бессмыденным матаном 
и никогда его больше не видеть. :(
\begin{dmath*}
(\e ^{x } )' = \e ^{x } \cdot \operatorname{ln}(\e ) \cdot 1.00 .
\end{dmath*}
Согласно Википедии
\begin{dmath*}
(\operatorname{f}(\e ^{x } ) '')' = \operatorname{f}(\e ^{x } ) '''\cdot \e ^{x } \cdot \operatorname{ln}(\e ) \cdot 1.00 .
\end{dmath*}
Легко догадаться, что
\begin{dmath*}
(\operatorname{f}(\e ^{x } ) ''\cdot \e ^{x } )' = \operatorname{f}(\e ^{x } ) ''\cdot \e ^{x } \cdot \operatorname{ln}(\e ) \cdot 1.00 + \operatorname{f}(\e ^{x } ) '''\cdot \e ^{x } \cdot \operatorname{ln}(\e ) \cdot 1.00 \cdot \e ^{x } .
\end{dmath*}
Согласно Википедии
\begin{dmath*}
(\operatorname{f}(\e ^{x } ) ''\cdot \e ^{x } \cdot \e ^{x } )' = \operatorname{f}(\e ^{x } ) ''\cdot \e ^{x } \cdot \e ^{x } \cdot \operatorname{ln}(\e ) \cdot 1.00 + (\operatorname{f}(\e ^{x } ) ''\cdot \e ^{x } \cdot \operatorname{ln}(\e ) \cdot 1.00 + \operatorname{f}(\e ^{x } ) '''\cdot \e ^{x } \cdot \operatorname{ln}(\e ) \cdot 1.00 \cdot \e ^{x } )\cdot \e ^{x } .
\end{dmath*}
Нарисуем кишечную палочку
\begin{dmath*}
(x )' = 1.00 .
\end{dmath*}
Легко догадаться, что
\begin{dmath*}
(\e ^{x } )' = \e ^{x } \cdot \operatorname{ln}(\e ) \cdot 1.00 .
\end{dmath*}
Нарисуем кишечную палочку
\begin{dmath*}
(x )' = 1.00 .
\end{dmath*}
Ах если бы сложение стало делением, \\ 
А, может, даже умножением! \\ 
Вот было бы прекрасно,\\ 
Если бы муссонные ветра дули не напрасно. \\ 
Тогда может быть и был бы смысл в этой жизни,\\ 
И равен бы он был морали этих строк, которой нет...\\ 
Так о чём же это я? Да ни о чём, собственно говоря.
Я просто хочу удалить этот файл с бессмыденным матаном 
и никогда его больше не видеть. :(
\begin{dmath*}
(\e ^{x } )' = \e ^{x } \cdot \operatorname{ln}(\e ) \cdot 1.00 .
\end{dmath*}
А что если Господь создал этот мир только для того, 
чтобы мы посчитали за него следующую производную:
\begin{dmath*}
(\operatorname{f}(\e ^{x } ) ')' = \operatorname{f}(\e ^{x } ) ''\cdot \e ^{x } \cdot \operatorname{ln}(\e ) \cdot 1.00 .
\end{dmath*}
Ах если бы сложение стало делением, \\ 
А, может, даже умножением! \\ 
Вот было бы прекрасно,\\ 
Если бы муссонные ветра дули не напрасно. \\ 
Тогда может быть и был бы смысл в этой жизни,\\ 
И равен бы он был морали этих строк, которой нет...\\ 
Так о чём же это я? Да ни о чём, собственно говоря.
Я просто хочу удалить этот файл с бессмыденным матаном 
и никогда его больше не видеть. :(
\begin{dmath*}
(\operatorname{f}(\e ^{x } ) '\cdot \e ^{x } )' = \operatorname{f}(\e ^{x } ) '\cdot \e ^{x } \cdot \operatorname{ln}(\e ) \cdot 1.00 + \operatorname{f}(\e ^{x } ) ''\cdot \e ^{x } \cdot \operatorname{ln}(\e ) \cdot 1.00 \cdot \e ^{x } .
\end{dmath*}
Согласно Википедии
\begin{dmath*}
(\operatorname{f}(\e ^{x } ) '\cdot \e ^{x } + \operatorname{f}(\e ^{x } ) ''\cdot \e ^{x } \cdot \e ^{x } )' = \operatorname{f}(\e ^{x } ) '\cdot \e ^{x } \cdot \operatorname{ln}(\e ) \cdot 1.00 + \operatorname{f}(\e ^{x } ) ''\cdot \e ^{x } \cdot \operatorname{ln}(\e ) \cdot 1.00 \cdot \e ^{x } + \operatorname{f}(\e ^{x } ) ''\cdot \e ^{x } \cdot \e ^{x } \cdot \operatorname{ln}(\e ) \cdot 1.00 + (\operatorname{f}(\e ^{x } ) ''\cdot \e ^{x } \cdot \operatorname{ln}(\e ) \cdot 1.00 + \operatorname{f}(\e ^{x } ) '''\cdot \e ^{x } \cdot \operatorname{ln}(\e ) \cdot 1.00 \cdot \e ^{x } )\cdot \e ^{x } .
\end{dmath*}
Нарисуем кишечную палочку
\begin{dmath*}
(x )' = 1.00 .
\end{dmath*}
Производная любого числа равна числу, не существующему в природе.
\begin{dmath*}
(2.00 )' = 0.00 .
\end{dmath*}
А что если Господь создал этот мир только для того, 
чтобы мы посчитали за него следующую производную:
\begin{dmath*}
(2.00 \cdot x )' = 2.00 \cdot 1.00 + 0.00 \cdot x .
\end{dmath*}
Нарисуем кишечную палочку
\begin{dmath*}
(x )' = 1.00 .
\end{dmath*}
Тривиальный переход ~--- это всё, на что мы способны
\begin{dmath*}
(2.00 )' = 0.00 .
\end{dmath*}
Согласно Википедии
\begin{dmath*}
(2.00 \cdot x )' = 2.00 \cdot 1.00 + 0.00 \cdot x .
\end{dmath*}
Притворимся, что знаем матан))
\begin{dmath*}
(x )' = 1.00 .
\end{dmath*}
Согласно Википедии
\begin{dmath*}
(x ^{2.00 } )' = (2.00 - 1.00 )\cdot x ^{2.00 - 1.00 } \cdot 1.00 .
\end{dmath*}
Производная любого числа равна числу, не существующему в природе.
\begin{dmath*}
(2.00 )' = 0.00 .
\end{dmath*}
Согласно Википедии
\begin{dmath*}
(2.00 \cdot x ^{2.00 } )' = 2.00 \cdot (2.00 - 1.00 )\cdot x ^{2.00 - 1.00 } \cdot 1.00 + 0.00 \cdot x ^{2.00 } .
\end{dmath*}
В СССР дети знали это с детского сада!
\begin{dmath*}
(\operatorname{cos}(2.00 \cdot x ^{2.00 } ) )' = ( -\operatorname{sin}(2.00 \cdot x ^{2.00 } ) ) \cdot (2.00 \cdot (2.00 - 1.00 )\cdot x ^{2.00 - 1.00 } \cdot 1.00 + 0.00 \cdot x ^{2.00 } ).
\end{dmath*}
А что если Господь создал этот мир только для того, 
чтобы мы посчитали за него следующую производную:
\begin{dmath*}
(\operatorname{cos}(2.00 \cdot x ^{2.00 } ) \cdot 2.00 \cdot x )' = \operatorname{cos}(2.00 \cdot x ^{2.00 } ) \cdot (2.00 \cdot 1.00 + 0.00 \cdot x )+ ( -\operatorname{sin}(2.00 \cdot x ^{2.00 } ) ) \cdot (2.00 \cdot (2.00 - 1.00 )\cdot x ^{2.00 - 1.00 } \cdot 1.00 + 0.00 \cdot x ^{2.00 } )\cdot 2.00 \cdot x .
\end{dmath*}
Легко догадаться, что
\begin{dmath*}
(-(\operatorname{cos}(2.00 \cdot x ^{2.00 } ) \cdot 2.00 \cdot x ))' = -(\operatorname{cos}(2.00 \cdot x ^{2.00 } ) \cdot (2.00 \cdot 1.00 + 0.00 \cdot x )+ ( -\operatorname{sin}(2.00 \cdot x ^{2.00 } ) ) \cdot (2.00 \cdot (2.00 - 1.00 )\cdot x ^{2.00 - 1.00 } \cdot 1.00 + 0.00 \cdot x ^{2.00 } )\cdot 2.00 \cdot x ).
\end{dmath*}
Легко догадаться, что
\begin{dmath*}
(( -(\operatorname{cos}(2.00 \cdot x ^{2.00 } ) \cdot 2.00 \cdot x )) \cdot 2.00 \cdot x )' = ( -(\operatorname{cos}(2.00 \cdot x ^{2.00 } ) \cdot 2.00 \cdot x )) \cdot (2.00 \cdot 1.00 + 0.00 \cdot x )+ ( -(\operatorname{cos}(2.00 \cdot x ^{2.00 } ) \cdot (2.00 \cdot 1.00 + 0.00 \cdot x )+ ( -\operatorname{sin}(2.00 \cdot x ^{2.00 } ) ) \cdot (2.00 \cdot (2.00 - 1.00 )\cdot x ^{2.00 - 1.00 } \cdot 1.00 + 0.00 \cdot x ^{2.00 } )\cdot 2.00 \cdot x )) \cdot 2.00 \cdot x .
\end{dmath*}
Чему равна производная числа??? НОЛЬ, НОЛЬ, НОЛЬ!!!
\begin{dmath*}
(2.00 )' = 0.00 .
\end{dmath*}
Притворимся, что знаем матан))
\begin{dmath*}
(x )' = 1.00 .
\end{dmath*}
Согласно Википедии
\begin{dmath*}
(x ^{2.00 } )' = (2.00 - 1.00 )\cdot x ^{2.00 - 1.00 } \cdot 1.00 .
\end{dmath*}
Тривиальный переход ~--- это всё, на что мы способны
\begin{dmath*}
(2.00 )' = 0.00 .
\end{dmath*}
А что если Господь создал этот мир только для того, 
чтобы мы посчитали за него следующую производную:
\begin{dmath*}
(2.00 \cdot x ^{2.00 } )' = 2.00 \cdot (2.00 - 1.00 )\cdot x ^{2.00 - 1.00 } \cdot 1.00 + 0.00 \cdot x ^{2.00 } .
\end{dmath*}
В СССР дети знали это с детского сада!
\begin{dmath*}
(\operatorname{sin}(2.00 \cdot x ^{2.00 } ) )' = \operatorname{cos}(2.00 \cdot x ^{2.00 } ) \cdot (2.00 \cdot (2.00 - 1.00 )\cdot x ^{2.00 - 1.00 } \cdot 1.00 + 0.00 \cdot x ^{2.00 } ).
\end{dmath*}
Легко догадаться, что
\begin{dmath*}
(-\operatorname{sin}(2.00 \cdot x ^{2.00 } ) )' = -(\operatorname{cos}(2.00 \cdot x ^{2.00 } ) \cdot (2.00 \cdot (2.00 - 1.00 )\cdot x ^{2.00 - 1.00 } \cdot 1.00 + 0.00 \cdot x ^{2.00 } )).
\end{dmath*}
А что если Господь создал этот мир только для того, 
чтобы мы посчитали за него следующую производную:
\begin{dmath*}
(( -\operatorname{sin}(2.00 \cdot x ^{2.00 } ) ) \cdot 2.00 )' = ( -\operatorname{sin}(2.00 \cdot x ^{2.00 } ) ) \cdot 0.00 + ( -(\operatorname{cos}(2.00 \cdot x ^{2.00 } ) \cdot (2.00 \cdot (2.00 - 1.00 )\cdot x ^{2.00 - 1.00 } \cdot 1.00 + 0.00 \cdot x ^{2.00 } ))) \cdot 2.00 .
\end{dmath*}
А что если Господь создал этот мир только для того, 
чтобы мы посчитали за него следующую производную:
\begin{dmath*}
(( -\operatorname{sin}(2.00 \cdot x ^{2.00 } ) ) \cdot 2.00 + ( -(\operatorname{cos}(2.00 \cdot x ^{2.00 } ) \cdot 2.00 \cdot x )) \cdot 2.00 \cdot x )' = ( -\operatorname{sin}(2.00 \cdot x ^{2.00 } ) ) \cdot 0.00 + ( -(\operatorname{cos}(2.00 \cdot x ^{2.00 } ) \cdot (2.00 \cdot (2.00 - 1.00 )\cdot x ^{2.00 - 1.00 } \cdot 1.00 + 0.00 \cdot x ^{2.00 } ))) \cdot 2.00 + ( -(\operatorname{cos}(2.00 \cdot x ^{2.00 } ) \cdot 2.00 \cdot x )) \cdot (2.00 \cdot 1.00 + 0.00 \cdot x )+ ( -(\operatorname{cos}(2.00 \cdot x ^{2.00 } ) \cdot (2.00 \cdot 1.00 + 0.00 \cdot x )+ ( -\operatorname{sin}(2.00 \cdot x ^{2.00 } ) ) \cdot (2.00 \cdot (2.00 - 1.00 )\cdot x ^{2.00 - 1.00 } \cdot 1.00 + 0.00 \cdot x ^{2.00 } )\cdot 2.00 \cdot x )) \cdot 2.00 \cdot x .
\end{dmath*}
Легко догадаться, что
\begin{dmath*}
(( -\operatorname{sin}(2.00 \cdot x ^{2.00 } ) ) \cdot 2.00 + ( -(\operatorname{cos}(2.00 \cdot x ^{2.00 } ) \cdot 2.00 \cdot x )) \cdot 2.00 \cdot x + \operatorname{f}(\e ^{x } ) '\cdot \e ^{x } + \operatorname{f}(\e ^{x } ) ''\cdot \e ^{x } \cdot \e ^{x } )' = ( -\operatorname{sin}(2.00 \cdot x ^{2.00 } ) ) \cdot 0.00 + ( -(\operatorname{cos}(2.00 \cdot x ^{2.00 } ) \cdot (2.00 \cdot (2.00 - 1.00 )\cdot x ^{2.00 - 1.00 } \cdot 1.00 + 0.00 \cdot x ^{2.00 } ))) \cdot 2.00 + ( -(\operatorname{cos}(2.00 \cdot x ^{2.00 } ) \cdot 2.00 \cdot x )) \cdot (2.00 \cdot 1.00 + 0.00 \cdot x )+ ( -(\operatorname{cos}(2.00 \cdot x ^{2.00 } ) \cdot (2.00 \cdot 1.00 + 0.00 \cdot x )+ ( -\operatorname{sin}(2.00 \cdot x ^{2.00 } ) ) \cdot (2.00 \cdot (2.00 - 1.00 )\cdot x ^{2.00 - 1.00 } \cdot 1.00 + 0.00 \cdot x ^{2.00 } )\cdot 2.00 \cdot x )) \cdot 2.00 \cdot x + \operatorname{f}(\e ^{x } ) '\cdot \e ^{x } \cdot \operatorname{ln}(\e ) \cdot 1.00 + \operatorname{f}(\e ^{x } ) ''\cdot \e ^{x } \cdot \operatorname{ln}(\e ) \cdot 1.00 \cdot \e ^{x } + \operatorname{f}(\e ^{x } ) ''\cdot \e ^{x } \cdot \e ^{x } \cdot \operatorname{ln}(\e ) \cdot 1.00 + (\operatorname{f}(\e ^{x } ) ''\cdot \e ^{x } \cdot \operatorname{ln}(\e ) \cdot 1.00 + \operatorname{f}(\e ^{x } ) '''\cdot \e ^{x } \cdot \operatorname{ln}(\e ) \cdot 1.00 \cdot \e ^{x } )\cdot \e ^{x } .
\end{dmath*}


Благодаря интуиции поймём, что
 \begin{dmath*}
( -\operatorname{sin}(2.00 \cdot x ^{2.00 } ) ) \cdot 0.00 + ( -(\operatorname{cos}(2.00 \cdot x ^{2.00 } ) \cdot (2.00 \cdot (2.00 - 1.00 )\cdot x ^{2.00 - 1.00 } \cdot 1.00 + 0.00 \cdot x ^{2.00 } ))) \cdot 2.00 + ( -(\operatorname{cos}(2.00 \cdot x ^{2.00 } ) \cdot 2.00 \cdot x )) \cdot (2.00 \cdot 1.00 + 0.00 \cdot x )+ ( -(\operatorname{cos}(2.00 \cdot x ^{2.00 } ) \cdot (2.00 \cdot 1.00 + 0.00 \cdot x )+ ( -\operatorname{sin}(2.00 \cdot x ^{2.00 } ) ) \cdot (2.00 \cdot (2.00 - 1.00 )\cdot x ^{2.00 - 1.00 } \cdot 1.00 + 0.00 \cdot x ^{2.00 } )\cdot 2.00 \cdot x )) \cdot 2.00 \cdot x + \operatorname{f}(\e ^{x } ) '\cdot \e ^{x } \cdot \operatorname{ln}(\e ) \cdot 1.00 + \operatorname{f}(\e ^{x } ) ''\cdot \e ^{x } \cdot \operatorname{ln}(\e ) \cdot 1.00 \cdot \e ^{x } + \operatorname{f}(\e ^{x } ) ''\cdot \e ^{x } \cdot \e ^{x } \cdot \operatorname{ln}(\e ) \cdot 1.00 + (\operatorname{f}(\e ^{x } ) ''\cdot \e ^{x } \cdot \operatorname{ln}(\e ) \cdot 1.00 + \operatorname{f}(\e ^{x } ) '''\cdot \e ^{x } \cdot \operatorname{ln}(\e ) \cdot 1.00 \cdot \e ^{x } )\cdot \e ^{x }  = ( -(\operatorname{cos}(2.00 \cdot x ^{2.00 } ) \cdot 2.00 \cdot x )) \cdot 2.00 + ( -(\operatorname{cos}(2.00 \cdot x ^{2.00 } ) \cdot 2.00 \cdot x )) \cdot 2.00 + (\operatorname{cos}(2.00 \cdot x ^{2.00 } ) \cdot 2.00 - ( -\operatorname{sin}(2.00 \cdot x ^{2.00 } ) ) \cdot 2.00 \cdot x \cdot 2.00 \cdot x )\cdot 2.00 \cdot x + \operatorname{f}(\e ^{x } ) '\cdot \e ^{x } + \operatorname{f}(\e ^{x } ) ''\cdot \e ^{x } \cdot \e ^{x } + \operatorname{f}(\e ^{x } ) ''\cdot \e ^{x } \cdot \e ^{x } + (\operatorname{f}(\e ^{x } ) ''\cdot \e ^{x } + \operatorname{f}(\e ^{x } ) '''\cdot \e ^{x } \cdot \e ^{x } )\cdot \e ^{x }  .
\end{dmath*}

Ух... Наконец-то мы сделали это!

\section{}
Таким образом, разложив функцию по Тейлору в точке $0.00$, получим следующее:
\begin{dmath*}\operatorname{cos}(2.00 \cdot x ^{2.00 } ) + \operatorname{f}(\e ^{x } )  = 1.00 + \operatorname{f}(1.00 ) + \frac{\operatorname{f}(1.00 ) '}{1!} \cdot (x - 0.00)^1 + \frac{\operatorname{f}(1.00 ) '+ \operatorname{f}(1.00 ) ''}{2!} \cdot (x - 0.00)^2 + \frac{\operatorname{f}(1.00 ) '+ \operatorname{f}(1.00 ) ''+ \operatorname{f}(1.00 ) ''+ \operatorname{f}(1.00 ) ''+ \operatorname{f}(1.00 ) '''}{3!} \cdot (x - 0.00)^3 + o((x - 0.00)^3)\end{dmath*}

Вот такая вот унылая фигня у нас получилась.

Спасибо за внимание))
\end{document}
